% Author: Henry Burgess
\documentclass{article}

% Setup packages
\usepackage{hyperref}
\usepackage{xcolor}
\usepackage{geometry}
\geometry{letterpaper, portrait, margin=0.6in}

% Paragraph spacing
\setlength{\parindent}{0pt}
\setlength{\parskip}{2pt}

\begin{document}
  %  Header section with name and contact details
  {\Huge\textbf{\uppercase{Henry Burgess}}} \hfill \begin{minipage}{0.26\linewidth}

  \textbf{Phone:} +1 (314) 891-2285

  \textbf{Email:} \href{mailto:henryjburg@gmail.com}{henryjburg@gmail.com}

  \textbf{LinkedIn:} \href{https://www.linkedin.com/in/henryjburg/}{@henryjburg}

  \end{minipage}

  \medbreak

  Highly adaptable Software Engineer with over two years of full-stack experience, currently residing in St. Louis, Missouri.
  Sponsored to relocate from Australia and work for the Department of Neuroscience at Washington University School of Medicine in St. Louis under the direct leadership of Dr. Linda Richards, Department Chair.
  Working on advancing online behavioral and cognitive testing capabilities and large-scale scientific metadata management. Effective presentation and communcation skills across disciplines and in professional forums.
  Seeking a role delivering objectives towards positive societal development, allowing personal investment in professional and technical capabilities including the potential for graduate study within the United States.

  % Education
  \section*{\centering\uppercase{Education}}

  {\large\textbf{Bachelor of Engineering (Hons) (Software)}}

  \medbreak

  \textbf{The University of Queensland} \hfill \textit{Brisbane, Australia}

  \textit{January 2017 - June 2021}

  \textbf{GPA:} 6.67 (equiv. 3.85), Honors Class I

  \textbf{Thesis:} ``Implementation of Online Neuropsychological Tasks using JavaScript"

  \textbf{Thesis Supervision:} Linda Richards AO, FAA, FAHMS, PhD; Ryan Dean, PhD; Richard Thomas, MAppSc

  \textbf{Awards and Honors:} UQ Future Leader (Class of 2021), Hawken Scholar (2020), Dean's commendation for academic excellence (2019, 2021)

  \medbreak

  \textbf{Dalian Neusoft University of Information} \hfill \textit{Dalian, Liaoning province, China}

  \textit{June 2018 - July 2018}

  Awarded a travel grant to participate in an innovation and entrepreneurship program facilitated by the Australian Government's New Columbo Plan.

  % Professional Experience
  % Note: draw out domains of capability, technical domains, working capabilities (communication, solo developer end-to-end), interpersonal capabilities (public speaking, leading, engagement with stakeholders, conference presentations)
  \section*{\centering\uppercase{Professional Experience}}

  {\large\textbf{Software Engineer II - Washington University School of Medicine in St. Louis}}

  \textit{September 2021 - Present \hfill St. Louis, MO, USA}

  Relocated to St. Louis to commence a new role in the Brain Development and Disorders Laboratory under the leadership of Dr. Linda Richards, Chair of the Department of Neuroscience and Edison Professor of Neuroscience at Washington University School of Medicine. Delivered experiments and tools to advance neuroscientific investigation of human decision-making and metacognition within a collaborative project between Dr. Richards and Dr. Peter Dayan FRS. Responsible for problem-solving in unfamiliar contexts while implementing novel solutions to address limitations in online behavioral and cognitive experiments. Gained expertise beyond undergraduate studies in frontend development tools such as TypeScript, React, and Webpack. Software development workflow entirely self-managed as the sole software engineer working alongside a team of neuroscientists. Developed research skills and familiarity with scientific communication and processes, HIPAA, and FAIR data principles.

  \medbreak

  {\large\textbf{Research Assistant - Queensland Brain Institute}}

  \textit{January 2021 - September 2021 \hfill Brisbane, Australia}

  Implemented three cognitive research tasks using JavaScript and jsPsych. Promoted to assist with task delivery and supervised participants attending the Australian Disorders of the Corpus Callosum (AusDoCC) 2021 conference.

  \medbreak

  {\large\textbf{Software Intern - Deswik Mining Consultants}}

  \textit{January 2020 - February 2020 \hfill Brisbane, Australia}

  Delivered bug fixes, UI enhancements, and maintenance in the \textit{Deswik.Sched} product development team using Visual Studio 2019 and C\#. Worked in an Agile environment, participated in daily stand-up meetings and sprint retrospectives, and used Atlassian's Confluence and Jira to manage workflow.

  \medbreak

  {\large\textbf{CSIRO (Research Assistant)}}

  \textit{June 2019 - July 2019 \hfill Brisbane, Australia}

  Developed a prototype geospatial web application using JavaScript and Google satellite imagery. Required to understand an agricultural context and UX requirements of end-users from subject-matter experts.

  \pagebreak

  {\large\textbf{Teaching Assistant - The University of Queensland}}

  \textit{January 2019 - June 2022 \hfill Brisbane, Australia}

  Guided student learning

  \textbf{CSSE1001 (Introduction to Software Engineering):} Python; Object-Oriented Programming (OOP)

  \textbf{CSSE3012 (The Software Process):} Software Development Life Cycle (SDLC); Agile

  \textbf{COMP4500 (Advanced Algorithms and Data Structures):} Java; Computer Science; Data Structures

  \textbf{DECO2800 (Design Computing Studio 2):} Java; Project Management; CI/CD

  % Skills and Competencies
  \section*{\centering\uppercase{Skills and Competencies}}

  {\large\textbf{Technical Skills}}

  Modern front-end development | REST APIs and back-end development | CI/CD workflow management | FAIR data principles

  \medbreak

  {\large\textbf{Professional Skill}}

  Cross-specialty communication | Advanced and novel problem solving | Teamwork | Software development project management

  % Projects
  \section*{\centering\uppercase{Projects}}

  \textbf{Metadata Aggregator for Reproducible Science (MARS)} \hfill \href{https://github.com/Brain-Development-and-Disorders-Lab/mars}{GitHub}

  Open-source metadata management web application that encourages FAIR data principles by ensuring lab-generated metadata is indexable and accessible.

  \textit{Tools: React, TypeScript, Webpack, Node.js, Express.js, MongoDB, Docker}

  \medbreak

  \textbf{Dynamic Cognitive Tasks} \hfill \href{https://github.com/Brain-Development-and-Disorders-Lab/mars}{GitHub}

  Architecture to support advanced computations or modeling for primarily online cognitive tasks, facilitating dynamic behavior and responses to participant input.

  \textit{Tools: Docker, R, MATLAB}

  \medbreak

  \textbf{jspsych-attention-check} \hfill \href{https://github.com/Brain-Development-and-Disorders-Lab/jspsych-attention-check}{GitHub}

  Implemented a jsPsych plugin using TypeScript to present attention-checks to participants completing behavioral and cognitive tasks online, improving data quality and reproducibility from online research.

  \textit{Tools: jsPsych, TypeScript, Webpack}

  \medbreak

  \textbf{Neurocog.js} \hfill \href{https://github.com/Brain-Development-and-Disorders-Lab/Neurocog.js}{GitHub}

  Developed JavaScript package augmenting the functionality of a jsPsych-based behavioral or cognitive task. Facilitates task integration with online platforms and aims to streamline developer and researcher experience when deploying research tasks online.

  \textit{Tools: jsPsych, TypeScript, Jest, Webpack}

  \pagebreak

  % Publications
  \section*{\centering\uppercase{Publications}}

  {\large\textbf{Peer-reviewed}}

  Richards, L. J., Barnby, J., Dean, R., \textbf{Burgess, H.}, Kim, J., Teunisse, A., ... \& Dayan, P. (2021). Increased persuadability and credulity in people with corpus callosum dysgenesis. \textit{Cortex}.
  \href{https://doi.org/10.1101/2021.12.28.21268413}{https://doi.org/10.1101/2021.12.28.21268413}

  \medbreak

  {\large\textbf{Conference Presentations}}

  \textbf{Cognitive Neuroscience Society Meeting}

  ``Realizing Dynamic Cognitive Tasks with Cloud-based Computation" \hfill \textit{2023}

  \medbreak

  \textbf{Society for Neuroscience}

  ``Neurocog.js; A new tool for running cognitive experiments in both lab and online environments." \hfill \textit{2022}

  \medbreak

  \textbf{IRC\textsuperscript{5} Meeting}

  ``Enabling behavioural research with Computer Science" \hfill \textit{2022}

  % Memberships
  \section*{\centering\uppercase{Memberships}}

  {\textbf{Society for Neuroscience}}

  \textit{Regular Member \hfill 2022 - Present}

  \medbreak

  {\textbf{Cognitive Neuroscience Society}}

  \textit{Graduate Member \hfill 2022 - Present}

  \medbreak

  {\textbf{The United States Research Software Engineer Association (US-RSE)}}

  \textit{Member \hfill 2022 - Present}

  \medbreak

  {\textbf{International Research Consortium for the Corpus Callosum and Cerebral Connectivity (IRC\textsuperscript{5})}}

  \textit{Associate member, Neuropsychology \hfill 2021 - Present}

  \medbreak

  {\textbf{Engineers Australia}}

  \textit{Graduate Member \hfill 2021 - Present}

  % Personal Life
  \section*{\centering\uppercase{Personal Life}}

  Outside of work, I appreciate a variety of outdoor hobbies including backpacking and running. While in the USA, I have taken the opportunity to explore Yosemite and Joshua Tree National Parks and have hiked to the summit of Mt. Whitney (14,505ft). Completing my first half-marathon in 2022, I enjoy running both as a hobby and stress-reliever around the expansive Forest Park in my current home of St. Louis. Cooking is an everyday hobby that I enjoy after returning home from work, and I often enjoy the opportunity to cook and host for others.

\end{document}
