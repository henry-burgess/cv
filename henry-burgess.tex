% Author: Henry Burgess
\documentclass{article}

% Setup packages
\usepackage{hyperref}
\usepackage{xcolor}
\usepackage{geometry}
\geometry{letterpaper, portrait, margin=0.6in}

% Paragraph spacing
\setlength{\parindent}{0pt}
\setlength{\parskip}{2pt}

\begin{document}
  %  Header section with name and contact details
  {\Huge\textbf{\uppercase{Henry Burgess}}} \hfill \begin{minipage}{0.26\linewidth}

  \textbf{Phone:} +1 (314) 891-2285

  \textbf{Email:} \href{mailto:henryjburg@gmail.com}{\color{blue}\underline{henryjburg@gmail.com}}

  \textbf{LinkedIn:} \href{https://www.linkedin.com/in/henryjburg/}{\color{blue}\underline{@henryjburg}}

  \end{minipage}

  \medbreak

  Highly adaptable Software Engineer with over two years of full-stack experience, currently residing in St. Louis, Missouri.
  Sponsored to relocate from Australia and work for the Department of Neuroscience at Washington University School of Medicine in St. Louis under the direct leadership of Dr. Linda Richards, Department Chair.
  Working on advancing online behavioral and cognitive testing capabilities and large-scale scientific metadata management. Effective presentation and communcation skills across disciplines and in professional forums.
  Seeking a role delivering objectives towards positive societal development, allowing personal investment in professional and technical capabilities including the potential for graduate study within the United States.

  % Education
  \section*{\centering\uppercase{Education}}

  {\large\textbf{Bachelor of Engineering (Hons.) (Software)}}

  \medbreak

  \textbf{The University of Queensland} \hfill \textit{Brisbane, Australia}

  \textit{January 2017 - June 2021}

  \textbf{GPA:} 6.67 (equiv. 3.85), Honors Class I

  \textbf{Thesis:} ``Implementation of Online Neuropsychological Tasks using JavaScript"

  \textbf{Thesis Supervision:} Linda Richards AO, FAA, FAHMS, PhD; Ryan Dean, PhD; Richard Thomas, MAppSc

  \textbf{Awards and Honors:} UQ Future Leader (Class of 2021), Hawken Scholar (2020), Dean's commendation for academic excellence (2019, 2021)

  \medbreak

  \textbf{Dalian Neusoft University of Information} \hfill \textit{Dalian, China}

  \textit{June 2018 - July 2018}

  Awarded a travel grant to participate in an innovation and entrepreneurship program facilitated by the Australian Government's New Columbo Plan.

  % Professional Experience
  % Note: draw out domains of capability, technical domains, working capabilities (communication, solo developer end-to-end), interpersonal capabilities (public speaking, leading, engagement with stakeholders, conference presentations)
  \section*{\centering\uppercase{Professional Experience}}

  {\large\textbf{Washington University School of Medicine in St. Louis \hfill Software Engineer II}}

  \textit{September 2021 - Present \hfill St. Louis, United States}

  Member of the \href{https://sites.wustl.edu/richardslab/}{\color{blue}\underline{\textit{Brain Development and Disorders Lab}}}, responsible for entire software development life cycle, including cross-disciplinary communication and project management.
  Primarily full-stack development activities utilizing tools such as TypeScript, React, and Webpack.
  Self-directed learning to acquire core technical competencies to deliver objectives.
  Engaged in collaborations with internationally recognized experts including Dr. Peter Dayan {\small{FRS}}.
  Successfully delivered projects advancing capabilities of online behavioral and cognitive research tasks, resulting in multiple conference presentations.
  Additional responsibilities include large-scale metadata management, data management policy, and social media management.
  Highly familiar with scientific communication and collaboration.

  \medbreak

  {\large\textbf{The University of Queensland \hfill Teaching Assistant}}

  \textit{January 2019 - June 2022 \hfill Brisbane, Australia}

  Guided student learning, marked student assessment, and collaborated on coursework development. Engaged for multiple semesters at the request of course cooridnators.

  \textbf{CSSE1001 (Introduction to Software Engineering):} Python; Object-Oriented Programming (OOP)

  \textbf{CSSE3012 (The Software Process):} Software Development Life Cycle (SDLC); Agile

  \textbf{COMP4500 (Advanced Algorithms and Data Structures):} Java; Computer Science; Data Structures

  \textbf{DECO2800 (Design Computing Studio 2):} Java; Project Management; CI/CD

  \medbreak

  {\large\textbf{Queensland Brain Institute \hfill Research Assistant}}

  \textit{January 2021 - September 2021 \hfill Brisbane, Australia}

  Promoted to Research Assistant while completing undergraduate project. Implemented three cognitive research tasks using JavaScript and jsPsych. Assisted with task delivery and supervision of participants attending the 2021 Australian Disorders of the Corpus Callosum (AusDoCC) conference.

  \pagebreak

  {\large\textbf{Deswik (Sandvik Group Member) \hfill Software Intern}}

  \textit{January 2020 - February 2020 \hfill Brisbane, Australia}

  Delivered bug fixes, interface enhancements, and general maintenance in the \href{https://www.deswik.com/product-detail/deswik-scheduler/}{\color{blue}\underline{\textit{Deswik.Sched}}} product development team. Used Visual Studio 2019 and C\# in an Agile environment, participated in daily stand-up meetings and sprint retrospectives. Used Atlassian's Confluence and Jira to manage workflow.

  \medbreak

  {\large\textbf{CSIRO \hfill Research Assistant}}

  \textit{June 2019 - July 2019 \hfill Brisbane, Australia}

  Developed geospatial web application prototype using JavaScript and Google satellite imagery. Required to understand an agricultural context and UX requirements of end-users from subject-matter experts.

  % Projects
  \section*{\centering\uppercase{Projects}}

  \textbf{Metadata Aggregator for Reproducible Science (MARS)} \hfill \href{https://github.com/Brain-Development-and-Disorders-Lab/mars}{\color{blue}\underline{GitHub}}

  Open-source metadata management web application that encourages FAIR data principles by ensuring lab-generated metadata is indexable and accessible.

  \textit{Tools: React, TypeScript, Webpack, Node.js, Express.js, MongoDB, Docker}

  \medbreak

  \textbf{Dynamic Cognitive Tasks} \hfill \href{https://github.com/Brain-Development-and-Disorders-Lab/mars}{\color{blue}\underline{GitHub}}

  Architecture to support advanced computations or modeling for primarily online cognitive research tasks, facilitating dynamic behavior and responses to participant input.

  \textit{Tools: Docker, R, MATLAB}

  \medbreak

  \textbf{jspsych-attention-check} \hfill \href{https://github.com/Brain-Development-and-Disorders-Lab/jspsych-attention-check}{\color{blue}\underline{GitHub}}

  Implemented a jsPsych plugin using TypeScript to present attention-checks to participants completing behavioral and cognitive research tasks online, improving data quality and reproducibility from online research.

  \textit{Tools: jsPsych, TypeScript, Webpack}

  \medbreak

  \textbf{Neurocog.js} \hfill \href{https://github.com/Brain-Development-and-Disorders-Lab/Neurocog.js}{\color{blue}\underline{GitHub}}

  Developed JavaScript package augmenting the functionality of a jsPsych-based behavioral or cognitive research task. Facilitates task integration with online platforms and aims to streamline developer and researcher experience when deploying research tasks online.

  \textit{Tools: jsPsych, TypeScript, Jest, Webpack}

  % Publications
  \section*{\centering\uppercase{Publications}}

  {\large\textbf{Peer-reviewed}}

  Richards, L. J., Barnby, J., Dean, R., \textbf{Burgess, H.}, Kim, J., Teunisse, A., ... \& Dayan, P. (2021). Increased persuadability and credulity in people with corpus callosum dysgenesis. \textit{Cortex}.
  \href{https://doi.org/10.1101/2021.12.28.21268413}{\color{blue}\underline{https://doi.org/10.1101/2021.12.28.21268413}}

  \medbreak

  {\large\textbf{Conference Presentations}}

  \textbf{Cognitive Neuroscience Society Meeting}

  ``Realizing Dynamic Cognitive Tasks with Cloud-based Computation" \hfill \textit{2023}

  \medbreak

  \textbf{Society for Neuroscience}

  ``Neurocog.js; A new tool for running cognitive experiments in both lab and online environments." \hfill \textit{2022}

  \medbreak

  \textbf{IRC\textsuperscript{5} Meeting}

  ``Enabling behavioural research with Computer Science" \hfill \textit{2022}

  \pagebreak

  % Memberships
  \section*{\centering\uppercase{Memberships}}

  {\textbf{Society for Neuroscience}}

  \textit{Regular Member \hfill 2022 - Present}

  \medbreak

  {\textbf{Cognitive Neuroscience Society}}

  \textit{Graduate Member \hfill 2022 - Present}

  \medbreak

  {\textbf{The United States Research Software Engineer Association (US-RSE)}}

  \textit{Member \hfill 2022 - Present}

  \medbreak

  {\textbf{International Research Consortium for the Corpus Callosum and Cerebral Connectivity (IRC\textsuperscript{5})}}

  \textit{Associate member, Neuropsychology \hfill 2021 - Present}

  \medbreak

  {\textbf{Engineers Australia}}

  \textit{Graduate Member \hfill 2021 - Present}

  % Personal Life
  \section*{\centering\uppercase{Personal Life}}

  Outside of work, I appreciate a variety of outdoor hobbies including backpacking and running. While in the USA, I have taken the opportunity to explore Yosemite and Joshua Tree National Parks and have hiked to the summit of Mt. Whitney (14,505ft). Completing my first half-marathon in 2022, I enjoy running both as a hobby and stress-reliever around the expansive Forest Park in my current home of St. Louis. Cooking is an everyday hobby that I enjoy after returning home from work, and I often enjoy the opportunity to cook and host for others.

\end{document}
