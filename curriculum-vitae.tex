%% Author: Henry Burgess

\documentclass{article}

%% Setup packages
\usepackage{hyperref}
\usepackage{geometry}
\geometry{letterpaper, portrait, margin=0.6in}
\setlength{\parindent}{0pt}
\setlength{\parskip}{2pt}

\begin{document}
  %%  Header section with name and contact details
  %%  Name
  {\Huge\textbf{Henry Burgess}}

  %%  Contact details
  \section*{Contact Details}

  \textbf{Phone:} +1 (314) 891-2285

  \textbf{Email:} \href{mailto:henryjburg@gmail.com}{henryjburg@gmail.com}

  \textbf{LinkedIn:} \href{https://www.linkedin.com/in/henryjburg/}{@henryjburg}

  \textbf{GitHub:} \href{https://github.com/henry-burgess}{@henry-burgess}

  \textbf{Website:} \href{https://henryburgess.me/}{henryburgess.me}

  %% Education
  \section*{Education}

  {\large\textbf{Bachelor of Engineering (Software)}}

  \textbf{The University of Queensland}

  \textit{January 2017 - June 2021 \hfill Brisbane, Australia}

  \textbf{GPA:} 3.85, Honors Class I

  \textbf{Thesis:} ``Implementation of Online Neuropsychological Tasks using JavaScript"

  \textbf{Thesis Supervision:} Linda Richards AO, FAA, FAHMS, PhD; Ryan Dean, PhD; Richard Thomas, MAppSc

  \textbf{Awards and Honors:} UQ Future Leader (Class of 2021), Hawken Scholar (2020), Dean's commendation for academic excellence (2019, 2021)

  \medbreak

  {\large\textbf{Dalian Neusoft University of Information}}

  \textit{June 2018 - July 2018 \hfill Dalian, Liaoning province, China}

  Received a travel grant and participated in an innovation and entrepreneurship program facilitated by the Australian Government's New Columbo Plan.

  %% Professional Experience
  \section*{Professional Experience}

  {\large\textbf{Software Engineer II - Washington University School of Medicine in St. Louis}}

  \textit{September 2021 - Present \hfill St. Louis, MO, USA}

  Relocated to St. Louis to commence a new role in the Brain Development and Disorders Laboratory under the leadership of Dr. Linda Richards, Chair of the Department of Neuroscience and Edison Professor of Neuroscience at Washington University School of Medicine. Delivered experiments and tools to advance neuroscientific investigation of human decision-making and metacognition within a collaborative project between Dr. Richards and Dr. Peter Dayan FRS. Responsible for problem-solving in unfamiliar contexts while implementing novel solutions to address limitations in online behavioral and cognitive experiments. Gained expertise beyond undergraduate studies in frontend development tools such as TypeScript, React, and Webpack. Software development workflow entirely self-managed as the sole software engineer working alongside a team of neuroscientists. Developed research skills and familiarity with scientific communication and processes, HIPAA, and FAIR data principles.

  \medbreak

  {\large\textbf{Research Assistant - Queensland Brain Institute}}

  \textit{January 2021 - September 2021 \hfill Brisbane, Australia}

  Originally joining the Brain Development and Disorders Laboratory as a student, promoted to increased responsibilities. Implemented three multi-context human cognitive experiments using JavaScript and jsPsych. Assisted with experiment delivery and supervised participants attending the Australian Disorders of the Corpus Callosum (AusDoCC) 2021 conference. 

  \medbreak

  {\large\textbf{Software Intern - Deswik Mining Consultants}}

  \textit{January 2020 - February 2020 \hfill Brisbane, Australia}

  Worked in the \textit{Deswik.Sched} product development team delivering bug fixes, UI enhancements, and maintenance. Gained experience with Visual Studio 2019 and the C\# programming language. Worked in an Agile environment, participated in daily stand-up meetings and sprint retrospectives, and used Atlassian's Confluence and Jira to manage workflow.

  \medbreak

  {\large\textbf{CSIRO (Research Assistant)}}

  \textit{June 2019 - July 2019 \hfill Brisbane, Australia}

  Developed a prototype geospatial web application using JavaScript and Google satellite imagery. Required to understand an agricultural context and UX requirements of end-users from subject-matter experts.

  \medbreak

  {\large\textbf{Teaching Assistant - The University of Queensland}}

  \textit{January 2019 - June 2022 \hfill Brisbane, Australia}

  \textbf{Introduction to Software Engineering (CSSE1001):} Introduced students to Python and Object-Oriented Programming concepts.

  \textbf{The Software Process (CSSE3012):} Introduced students to the software development life cycle and Agile principles.

  \textbf{Advanced Algorithms and Data Structures (COMP4500):} Marked assessment of runtime complexity, computational complexity, and dynamic programming.

  \textbf{Design Computing Studio 2 (DECO2800):} Managed software teams of 6 students working on a large-scale Java software project using Git.

  %% Memberships
  \section*{Memberships}
  \large\textbf{Society for Neuroscience}

  \textit{Regular Member (2022 - Present)}

  \large\textbf{Cognitive Neuroscience Society}

  \textit{Graduate Member (2022 - Present)}

  \large\textbf{The United States Research Software Engineer Association}

  \textit{Member (2022 - Present)}  

  \large\textbf{International Research Consortium for the Corpus Callosum and Cerebral Connectivity (IRC\textsuperscript{5})}

  \textit{Associate member, Neuropsychology (2021 - Present)}

  \large\textbf{Engineers Australia}

  \textit{Graduate Member (2021 - Present)}

  %% Publications
  \section*{Publications}
  \large\textbf{General Audience} 
  \large\textit{2022} 
  \large\textit{Australian Disorders of the Corpus Callosum (AusDoCC) Newsletter} 
  Wrote an article ``Online Research and Web Accessibility" submitted and published to the ``From our researchers..." column of the May 2022 AusDoCC newsletter.   
  
                      
  \large\textit{2021} 
  \large\textit{NeuroDesk running on Azure}    
  Published a tutorial describing how researchers could utilize free computational resources offered by Microsoft Azure to run an instance of the containerized \textit{NeuroDesk} image processing operating system. 
  
                      
  \large\textbf{Peer-reviewed and Preprint} 
  \large\textit{2022}
  \textbf{Burgess, H.}, Dean, R., Thomas, R. N. \& Richards, L. J. (2022). Neurocog.js; A new tool for running jsPsych-based cognitive experiments in both lab and online environments. (In preparation). 


  \large\textit{2021} 
  Richards, L. J., Barnby, J., Dean, R., \textbf{Burgess, H.}, Kim, J., Teunisse, A., ... \& Dayan, P. (2021). Increased persuadability and credulity in people with corpus callosum dysgenesis. \textit{Cortex}. 
  https://doi.org/10.1101/2021.12.28.21268413 

  \large\textbf{Abstracts}
  \large\textit{2022}
  \large\textit{Cognitive Neuroscience Society Meeting, 2023} 
  ``Realizing Dynamic Cognitive Tasks with Cloud-based Computation"  
  

  \large\textit{Society for Neuroscience, Neuroscience 2022}
  ``Neurocog.js; A new tool for running cognitive experiments in both lab and online environments."

  %% Presentations
  \large\textbf{Presentations}
  \large\textbf{Scientific Audience}
  \large\textit{2022}
  \large\textit{IRC\textsuperscript{5} Meeting (Frisco, TX)}
  ``Enabling behavioural research with Computer Science"
\end{document}